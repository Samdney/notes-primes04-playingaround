%%%%%%%%%%%%%%%%%%%%%%%%%%%%%%%%%%%%%%%%%%%%%%%%%%%%%%%%%%
% preamble.tex
% template-preamble
%
% Author: Carolin Zöbelein
% Email: contact@carolin-zoebelein.de
% PGP: D4A7 35E8 D47F 801F 2CF6 2BA7 927A FD3C DE47 E13B
%%%%%%%%%%%%%%%%%%%%%%%%%%%%%%%%%%%%%%%%%%%%%%%%%%%%%%%%%%
\documentclass{scrartcl}

\usepackage[headsepline,footsepline]{scrpage2}
\pagestyle{scrheadings}
\clearscrheadfoot
\setheadsepline{1.5pt}
\setfootsepline{1.5pt}

% Head
\ohead{\headmark}
\automark{section}

% Footer
\cfoot{\pagemark}

\usepackage[utf8x]{inputenc}
\usepackage[english]{babel}

\usepackage{footnote}
\usepackage{amssymb}
\usepackage{url}
\usepackage{graphicx}
\usepackage{amsmath}
\usepackage{hyperref}
%\usepackage{minitoc}
\usepackage{float}
\usepackage{longtable}
\usepackage{enumitem}

\usepackage{verbatim}	% For block comments


% Font style from resume
\usepackage{lmodern}
\renewcommand*\familydefault{\sfdefault}
\usepackage{mathptmx}%% Font Times


%\usepackage{fontawesome}
\usepackage{fontawesome5}


\usepackage{listings}
\lstset{
language=Python,
basicstyle=\small\sffamily,
%basicstyle=\tiny\sffamily,
numbers=left,
numberstyle=\tiny,
frame=tb,
%frame=single,
columns=fullflexible,
showstringspaces=false
}

\newtheorem{theorem}{Theorem}[section]
\newtheorem{lemma}[theorem]{Lemma}
\newtheorem{definition}[theorem]{Definition}
\newtheorem{example}[theorem]{Example}
\newtheorem{xca}[theorem]{Exercise}
\newtheorem{remark}[theorem]{Remark}

% Example: \authoremail{example@example.com, AAAA BBBB CCCC DDDD EEEE FFFF GGGG HHHH IIII JJJJ}
\newcommand{\authoremail}[2]{\textit{E-mail address:} \texttt{#1}, \textit{PGP Fingerprint}: \texttt{#2}}

% Example: \authorurl{http://www.example.com}
\newcommand{\authorurl}[1]{\textit{URL:} \url{#1}}

% Example: \keywords{keyword1, keyword2, keyword3}
\newcommand{\keywords}[1]{\textbf{Keywords:} #1}

% Example: \license{example license}
\newcommand{\license}[1]{\textbf{License:} #1}

% Example: \subjclass{2010}{Mathematics Subject Classification}{Primary 11N05}
\newcommand{\subjclass}[3]{\textbf{Subjclass:} #1 \textit{#2}. #3.}

% ======================================================================
\begin{document}
% ======================================================================
\begin{titlepage}
	%\begin{flushleft}
	%	\includegraphics[width=0.15\textwidth]{example-image-1x1}\par\vspace{1cm}
	%\end{flushleft}
	\begin{center}
	%{\scshape\LARGE Pony University \par}
	%\vspace{1cm}
    \begin{center}
            \includegraphics[width=0.2\textwidth]{images/logo_web.png}
    \end{center}
    \vspace{1cm}
	{\scshape\Large Research notes\par}
	\vspace{1.5cm}
	{\huge\bfseries Primes (part 04): Playing around\par}
	\vspace{2cm}
	{\Large\itshape Carolin Z\"obelein\footnote{PGP signing key (NOT for communication!): 8F31 C7C6 E67E 9ACE 8E12 E2EF 0DE3 A4D3 BA87 2A8B}\par}
	\vfill
	%supervised by\par
	%Dr.~John \textsc{Doe}
	\textsc{id: notes\_0004}\par
	\textsc{License: CC-BY-NC-ND}
	\vfill

	\fbox{\begin{minipage}{30em}
	Carolin Zöbelein\\
	Independent mathematical scientist

	\vspace{0.3cm}
	E-Mail: contact@carolin-zoebelein.de\\
	PGP: D4A7 35E8 D47F 801F 2CF6 2BA7 927A FD3C DE47 E13B\\
	Website: \url{https://research.carolin-zoebelein.de}
	\end{minipage}}

	\vfill

% Bottom of the page
	{\large Version: v01\par}
	{\large May 21, 2019\par}
	\end{center}
\end{titlepage}

% Suppress page numbers
%\thispagestyle{empty}
\pagenumbering{gobble}
% ======================================================================
% ----------------------------------------------------------------------
\newpage
\section*{Abstract}
\label{s:abstract}
% ----------------------------------------------------------------------
\begin{abstract}
	Some playing around with different ways of intersection solutions.
\end{abstract}
% ----------------------------------------------------------------------
\section*{Content}
\label{s:content}
% ----------------------------------------------------------------------
\begin{enumerate}
	\item [I.] Page 1 - 6: Extenting from $x$ to $\left(2xy + x + y\right)$ as part of $k$ solution 
	\item [II.] Page 7 - 10: Checking for possible additional terms and factors of this extension
	\item [III.] Page 11 - 17: Looking at common and not common factor(s) case of original $n\left(2x + 1\right)$ equation odd numbers $\left(2x + 1\right)$
\end{enumerate}
% ======================================================================
\end{document}
